\documentclass[swedish]{article}
\usepackage[utf8]{inputenc}
\usepackage[swedish]{babel}
\usepackage{mathtoolsplus}
\usepackage{listings}  % For displaying LaTeX code

\title{MathtoolsPlus Package Documentation}
\author{Eduards Abisevs}
\date{\today}

\begin{document}

\maketitle
\tableofcontents

\section{Introduction}
This document provides an overview of the custom \texttt{mathtoolsplus} LaTeX package, designed to assist with creating math and physics documents. The package includes:
\begin{itemize}
    \item Custom theorem and example environments
    \item Boxed environments for problems and solutions
    \item Diagrams with \texttt{TikZ}
    \item Custom chapter and section formatting
\end{itemize}

\section{Theorem Environments}

\subsection{Output}

\begin{theorem}
Låt \( a \) och \( b \) vara reella tal. Om \( a = b \), så är \( a^2 = b^2 \).
\end{theorem}

\begin{lemma}
För alla reella tal \( x \) och \( y \), 
så gäller \( (x + y)^2 = x^2 + 2xy + y^2 \).
\end{lemma}

\begin{definition}
En \textit{funktion} är en relation mellan en mängd av indata och en mängd av möjliga utdata, där varje indata relateras till exakt ett utdata.
\end{definition}

\subsection{Code}

\begin{lstlisting}[language=TeX]
\begin{theorem}
Låt \( a \) och \( b \) vara reella tal.
Om \( a = b \), så är \( a^2 = b^2 \).
\end{theorem}

\begin{lemma}
För alla reella tal \( x \) och \( y \),
så gäller \( (x + y)^2 = x^2 + 2xy + y^2 \).
\end{lemma}

\begin{definition}
En \textit{funktion} är en relation mellan en mängd av indata och en mängd av möjliga utdata, där varje indata relateras till exakt ett utdata.
\end{definition}
\end{lstlisting}

\section{Example Environment}

\subsection{Output}

\begin{example}
Detta är ett grundläggande exempel utan ytterligare text.
\end{example}

\begin{example}[Lösa en ekvation]
Lös \( x^2 = 4 \). Lösningarna är \( x = 2 \) och \( x = -2 \).
\end{example}

\subsection{Code}

\begin{lstlisting}[language=TeX]
\begin{example}
Detta är ett grundläggande exempel utan ytterligare text.
\end{example}

\begin{example}[Lösa en ekvation]
Lös \( x^2 = 4 \). Lösningarna är \( x = 2 \) och \( x = -2 \).
\end{example}
\end{lstlisting}

\section{Boxed Environments for Problem Solving}

\subsection{Output}

\begin{given}
Vi har en massa \( m = 5 \, \mathrm{kg} \).
\end{given}

\begin{sought}
Vi söker kraften \( F \) givet att \( a = 2 \, \mathrm{m/s^2} \).
\end{sought}

\begin{solution}
Med hjälp av Newtons andra lag: \( F = ma = 5 \times 2 = 10 \, \mathrm{N} \).
\end{solution}

\begin{answer}
Kraften är \( F = 10 \, \mathrm{N} \).
\end{answer}

\subsection{Code}

\begin{lstlisting}[language=TeX]
\begin{given}
Vi har en massa \( m = 5 \, \mathrm{kg} \).
\end{given}

\begin{sought}
Vi söker kraften \( F \) givet att \( a = 2 \, \mathrm{m/s^2} \).
\end{sought}

\begin{solution}
Med hjälp av Newtons andra lag: \( F = ma = 5 \times 2 = 10 \, \mathrm{N} \).
\end{solution}

\begin{answer}
Kraften är \( F = 10 \, \mathrm{N} \).
\end{answer}
\end{lstlisting}

\section{Diagrams with TikZ}

\subsection{Output: Horizontal Layout}

\begin{given_diagram}[h]{
  \begin{tikzpicture}
    \draw[->] (0,0) -- (2,0) node[below] {$x$};
    \draw[->] (0,0) -- (0,2) node[left] {$y$};
    \draw (0,0) -- (1.5,1.5);
  \end{tikzpicture}
}{Detta är ett enkelt koordinatsystemdiagram.}
\end{given_diagram}

\subsection{Output: Vertical Layout}

\begin{given_diagram}[v]{
  \begin{tikzpicture}
    \draw (0,0) circle (1);
    \draw[->] (-1.5,0) -- (1.5,0) node[right] {$x$};
    \draw[->] (0,-1.5) -- (0,1.5) node[above] {$y$};
  \end{tikzpicture}
}{Detta är en diagram som visar enhetscirkeln.}
\end{given_diagram}

\subsection{Code}

\begin{lstlisting}[language=TeX]
\begin{given_diagram}[h]{
  \begin{tikzpicture}
    \draw[->] (0,0) -- (2,0) node[below] {$x$};
    \draw[->] (0,0) -- (0,2) node[left] {$y$};
    \draw (0,0) -- (1.5,1.5);
  \end{tikzpicture}
}{Detta är ett enkelt koordinatsystemdiagram.}
\end{given_diagram}

\begin{given_diagram}[v]{
  \begin{tikzpicture}
    \draw (0,0) circle (1);
    \draw[->] (-1.5,0) -- (1.5,0) node[right] {$x$};
    \draw[->] (0,-1.5) -- (0,1.5) node[above] {$y$};
  \end{tikzpicture}
}{Detta är en diagram som visar enhetscirkeln.}
\end{given_diagram}
\end{lstlisting}

\section{Switching to English}

\subsection{Output in English}

\selectlanguage{english}

\begin{theorem}
Let \( a \) and \( b \) be real numbers. If \( a = b \), then \( a^2 = b^2 \).
\end{theorem}

\begin{lemma}
For any real numbers \( x \) and \( y \), \( (x + y)^2 = x^2 + 2xy + y^2 \).
\end{lemma}

\begin{definition}
A \textit{function} is a relation between a set of inputs and a set of possible outputs, where each input is related to exactly one output.
\end{definition}

\subsection{Code in English}

\begin{lstlisting}[language=TeX]
\selectlanguage{english}

\begin{theorem}
Let \( a \) and \( b \) be real numbers. If \( a = b \), then \( a^2 = b^2 \).
\end{theorem}

\begin{lemma}
For any real numbers \( x \) and \( y \), \( (x + y)^2 = x^2 + 2xy + y^2 \).
\end{lemma}

\begin{definition}
A \textit{function} is a relation between a set of inputs and a set of possible outputs, where each input is related to exactly one output.
\end{definition}
\end{lstlisting}

\end{document}
§
